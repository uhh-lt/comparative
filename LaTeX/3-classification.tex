\chapter{Classification of Comparative Sentences}
\section{Description of the experiments}
The goal of this thesis is to train a machine learning model in a way that it can decide whether a sentence contains a comparison between two defined objects or not.

To achieve this goal, the task was formulated in three ways of different granularity. In all tasks, the machine learning model was provided with the sentence and the objects.

In the first experiment, the model was trained to classify a sentence into one of the four categories described above.
As stated below, the class OTHER adds uncertainty to the model. This problem is handled in next experiments. In the second experiment, all sentences with this class were removed prior training. In the third experiment, the class OTHER was joined with the class NONE.  Thus, in those experiments, the model has to decide between three classes.

In the fourth experiment, BETTER, WORSE and OTHER are joined to the class ARG while the fifth experiment joins BETTER and WORSE to ARG and OTHER with NONE. Thus, the fourth and fifth experiments are binary classification tasks.

\section{Evaluation}
The evaluation of the results was done with stratified k-fold cross-validation where k is 3. The overall F1 score of each fold is the weighted average of the F1 scores of each class, where the weights are the number of examples per class. It was produced using the \texttt{classification\_report} function of sklearn.
While discussing the features, single F1 scores are presented. Those scores are the unweighted average of the F1 scores of each fold.

Following \cite{Daxenberger2017What-is-the-Ess}, the results were also evaluated with training on two of the domains and evaluating on the leftover one.

\section{Algorithms}
The classification was performed with SkLearn (\cite{scikit-learn}).

\section{Features}
Several features and feature combinations were tested. The results are presented in table X and Y. As described above, the F1 scores in the tables are the unweighted averages of the three F1 scores of each fold.

Every feature was tested on different parts of the sentence: the whole sentence, all words before the first object, all words after the second object and all words between objects. In doing so, the objects either stayed in the sentence, were removed or replaced. Two different replacement strategies were tested: replacing both objects with OBJECT and replacing the first object with OBJECT\_A and the second object with OBJECT\_B. The replacement approach should test if the objects influence the decision of the classifier. For example, if "Python" is always the "better" object the classifier might become biased.

The following section only describes features which worked out well, leaving out bad combinations (for example, using only the first or last part was not helpful in most cases).\newline

In the first step, n-gram models where tested. Every uni-, bi and trigram in the whole training set was implemented as a binary feature. Restricting on frequency was not helpful. Also, trigrams were not helpful which can be explained by the length of the sentences.



\section{Results}

% Please add the following required packages to your document preamble:
% \usepackage{booktabs}
% Please add the following required packages to your document preamble:
% \usepackage{booktabs}
\begin{table}[h]
\centering
\caption{Results (3-fold; Linear Support Vector Machine)}
\label{tbl:results}
\begin{tabularx}{\textwidth}{lXXXXX}
\toprule
                            & Four classes &Three A & Three B & Binary A & Binary B \\ \midrule
Sentence Embeddings (M) &              &               &              &                &             \\
Unigrams (M)            &              &               &              &                &             \\ \midrule
                            &              &               &              &                &             \\
                            &              &               &              &                &             \\ \bottomrule
\end{tabularx}
\end{table}
\subsection{Four Labels}
\subsection{Three labels}
\subsection{Binary}

\section{Discussion}