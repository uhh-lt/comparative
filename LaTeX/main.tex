% ---------------------------------------------------------------------
% Das Dokument kompiliert mit pdflatex und ist auf Basis 
% von Koma-Script entstanden. 
%
% Autor des Templates (für Anmerkungen): 
% Michael von Riegen, riegen@informatik.uni-hamburg.de
%
% Einzelne Code-Teile für das Titelblatt sind aus dem Template 
% von Benjamin Kirchheim entnommen.
%
% 25.05.09, Frank Langanke: Vorlage auf aktuelle KOMA-Version aktualisiert
% 26.05.09, Michael von Riegen: Anmerkung --> aktuelles Koma-Script ist nötig!
% 17.10.2016 neues Uni logo
% ---------------------------------------------------------------------
\documentclass[11pt,DIV=15,BCOR=20mm,bibliography=totoc]{scrbook}

% Import von Paketen und Optionen die das gesamte Dokument betreffen
% sind in myPreamble.sty ausgelagert.
\usepackage{myPreamble}
   
% Arbeitet man nur an einem Kapitel, wird durch folgenden Befehl nur dieses eingebunden.
% Spart manuelles auskommentieren von vielen include-Befehlen;
% hat keine Auswirkung auf input-Befehle
% \includeonly{kapitel1}
   
\begin{document}

% TITELSEITE
\begin{titlepage}

	% Fehler "destination with the same identifier" unterdrücken...
  \setcounter{page}{-1}

	% Titelseite
	\begin{figure}[h]
		\begin{minipage}[b]{62mm}
			\includegraphics[width=62mm]{images/unilogo}
		\end{minipage}
		\hspace{4cm}
		\begin{minipage}[b]{59mm}
		%	\includegraphics[width=59mm]{images/minlogo}
		\end{minipage}
	\end{figure}

	\vfill
	
	\begin{center}
		% Diplomarbeit 
		\noindent { \huge
			Master Thesis \\
		}
		\vspace{14mm}
		% Titel
		\noindent \textbf{\huge
		  Comparative Argument Mining
		}
		\vspace{60mm}	
	\end{center}
	
	\vfill
	
	\noindent{\textbf{Mirco Franzek}} \\
	\noindent{\textrm{5franzek@informatik.uni-hamburg.de}} \\
	\noindent{\textrm{Studiengang Informatik}} \\
	\noindent{\textrm{Matr.-Nr. 6781911}} \\
	\begin{tabbing}
	\hspace{8em} \=  \kill
	Erstgutachter: \> Prof. Dr. Chris Biemann \\
	Zweitgutachter: \> Dr. Alexander Panchenko \\
	~ \\
	Abgabe: 23. Mai 2018
	\end{tabbing}
	
	% Rückseite der Titelseite mit Zitat
	\newpage 
	\thispagestyle{empty}
	\setcounter{page}{0}

	% wenn man Lust auf ein Zitat hat...
	% ... ansonsten auskommentieren
%	~\\ \vfill \noindent 
	
%I am C3P0, protocol droid, human-cyborg relations.\\I am fluent in over 6 million forms of communication.
%	\textit{-- C3PO}
\end{titlepage}



% VERZEICHNISSE (Inhaltsverzeichnis, Abkürzungen)
% Vorspann einleiten --> Seitennummerierung römisch
\frontmatter

% Inhaltsverzeichnis
\tableofcontents
\cleardoublepage

% Hauptteil einleiten --> Seitennummerierung wieder arabisch
\mainmatter

% -----------------------------------------------------------------------
% -----------------------------------------------------------------------
% -----------------------------------------------------------------------
% Einleitung
% -----------------------------------------------------------------------
% -----------------------------------------------------------------------
% -----------------------------------------------------------------------
\chapter{Introduction}

\section{Motivation: An Open-Domain Comparative Argumentative Machine (CAM)}


The topic of this master thesis is \emph{Argument Mining}, more precisely \emph{Comparative Argument Mining}. Argument Mining is an area in Natural Language Processing which gained popularity quite recently. The first workshop organized by the \emph{Association for Computational Linguistics} (ACL) was just hold three years ago, in 2014 (\cite{W14-21:2014}).\\
The aim of Argument Mining is the analysis of arguments in natural language. One can identify three main tasks: the identification of argumentative sentences, the components of an argument and their relations (cite). There are several definitions of the term \emph{argument} and it's structure (see section \ref{sec:argth}). In Argument Mining, the Claim/Premise model is often used. A claim is the statement of a sentence which is supported or attacked by one or more premises. An argument of this structure may look like the following:
\begin{quote}
    [X will win$]_{claim}$ [because$]_{support}$ [X can do Y$]_{premise}$
\end{quote}
The knowledge obtained by analyzing such sentences can be used \ldots\newline

Comparative Arguments are a special kind of arguments. The claim of a comparative argument is a comparison between one or more objects. The comparison frequently contains a direction (or polarity) in a sense that \emph{A is better / worse than B}.




\section{Applications}
\label{sec:applications}
A lot of

\section{Related Work}
\subsection{Argumentation Theory}

\cite{Habernal2014Argumentation-M} presented a comparison between the results of two different annotation studies. One used the Claim/Premise-Model, while the other one used the Toulmin model. They emphasized that there is no "one-size-fits-all" model.

\label{sec:argth}
\subsection{Argument Mining}
\label{sec:argmine}
To the present day, only a few papers on Comparative Argument Mining are published. Nevertheless, much work on a variety Argument Mining problems has been published.\newline

\cite{Lippi2016Argumentation-M} gives a summary of the research topic \emph{Argument Mining} in general. They introduce five dimensions to describe Argument Mining problems: granularity of input, the genre of input, argument model, the granularity of target and goal of analysis.  Furthermore, the typical steps of Argument Mining Systems are defined. First, the input must be divided into argumentative (e.g. claim and premise) and non-argumentative. This step is defined as a classification problem. Second, the boundaries of the argumentative units are identified; this is understood as a segmentation problem. Third, the relations between argumentative units are identified. For instance, claims and premises are connected with a \enquote{support} relation.\newline


A recent publication on Comparative Argument Mining is \cite{gupta2017identifying}, where a set of rules for the identification of comparative sentences (and the compared entities) is derived from \emph{Syntactic Parse Trees}. With this rules, the authors achieved an F1-Score of 0,87 for the identification of comparative sentences. However, the rules were obtained from 50 abstracts of biomedical papers. Such being the case, they are domain dependent. Also, comparisons are frequent in biomedical publications.\newline

Because this thesis deals with user-generated content from the web, publications dealing with similar data are of interest.

The challenges occurring while processing texts from social media are described in \cite{Snajder2017Social-Media-Ar}.  In this publication, social media is broadly defined as \enquote{less controlled communication environments [...]}. Besides the noisiness of text, missing argument structures and poorly formulated claims are mentioned. It is expected that the text used in this thesis will have the same shortcomings. Additionally, \cite{Snajder2017Social-Media-Ar} emphasized that analyzing social media texts can delivery reasons behind opinions. 

In addition to the challenges mentioned above, \cite{Dusmanu2017Argument-Mining} also points to the specialized jargon used in user-generated content like hashtags and emoticons. With this in mind, \cite{Dusmanu2017Argument-Mining} classified tweets about the \enquote{Brexit} and \enquote{Grexit} either as argumentative or as non-argumentative. Besides features used in other mentioned papers, features covering hashtags and sentiment are added. They achieve an F1-Score of 0.78 (Logistic Regression) for the classification. It needs to be said that the data set is small and the domain is rather specific.\newline

Many publications on argument mining are dealing with a classification problem of some kind. Publications dealing with the identification of argument structures are of relevance for this thesis.

\cite{Aker2017What-works-and-} summarized and compared features used in other publications for identification of argumentative sentences. In addition, a Convolutional Neural Network was tested. Two existing corpora and six different classification algorithms were used. As a result, structural features are most expressive; Random Forest is the best classifier.

\cite{Stab2014Identifying-Arg} described a two-step procedure to identify components or arguments (such as claim and premise) and their relationships (\enquote{premise A supports claim B}). The identification step is formulated as a multiclass classification. The features are examined for the classification task in this thesis. F1-Score.

The role of discourse markers in the identification of claims and premises are discussed in \cite{Eckle-Kohler2015On-the-Role-of-}. A discourse marker is a word or a phrase which connects discourse units (citation). For instance, the word "as" can show a relation between claim and premise: "As the students get frustrated, their performance generally does not improve".  A similar function for words like "better", "worse" or "because" is expected in this thesis. \cite{Eckle-Kohler2015On-the-Role-of-} showed that discourse markers are good at x, they obtain an F1-Score of y.

A summary of several features for the identification of argumentative sentences can be found in chapter \ref{sec:features}.

%To my knowledge, there is no work dealing with Comparative Argument Mining (CAM) without domain restrictions. However, there is a lot of work on Argument Mining in general.\newline
%
%Recent work in CAM was presented in \cite{gupta2017identifying}. In this paper, \emph{syntactic dependency graphs} were used to obtain patterns of comparative sentences. A restriction on the sentence was that both entities must be known (implicit comparisons are out of scope).  The rules were applied to the tasks of identifying comparative sentences (F1 score 0.87), the compared aspect (0.78), the scale indicator (like "lower"; F1 score 0.81 ) and the compared entities (0.77). The evaluation was performed on 189 comparisons extracted from 125 abstracts from papers of the bio-medical domain. \newline
%
%Work on identifying argumentative sentences, their components and relations is also beneficial to this thesis, as some features and methods might be reused. An overview of features for those tasks was given in \cite{Aker2017What-works-and-}. Features and their combinations which were used in several other papers published between 2011 and 2015 were evaluated by the authors. To perform the experiments, several classifiers, ranging from simple linear models as Logistic Regression to Convolutional Neural Networks were used. 
%Structural features were reported as most useful for argumentative sentence detection whereas \emph{syntactic features} were the least useful ones (in contrast to \cite{gupta2017identifying}, were syntactic dependency was used successfully). No classifier clearly outperformed all others, yet Random Forests were best in one corpus and comparable to the rest on the other one.\newline
%
%The concept of claims was analyzed for six data sets in \cite{Daxenberger2017What-is-the-Ess}. The intention was to test how different conceptualization of claims impairs their classification. In cross-domain experiments, classifiers were trained on one full dataset (source domain) and tested against the other datasets (target domain). Featured-based approaches with all features outperformed the deep learning approaches in most combinations. Additionally, in-domain experiments on one dataset (10-fold cross-validation) were conducted. Here, Logistic Regression with syntax features gave the best results. Also, lexical and embedding features were useful.\newline
%
%
%User-generated web discourse was used as the data source in this thesis. Therefore, it is valuable to review articles which handle this kind of data.
%Argument mining for twitter data is performed in \cite{Dusmanu2017Argument-Mining}. In addition to find argumentative tweets, the authors try to distinguish factual arguments from personal opinion and find the source for the first one. Since tweets are short and user-generated, this can provide some insights for this thesis. For the task of argument identification, a F1 score of 0.78 is reached using Logistic Regression. As in \cite{Daxenberger2017What-is-the-Ess}, \emph{lexical features} (unigrams, bigrams and WordNet verb synsets) where quite powerful: Logistic Regression with lexical features produced a F1 score of 0.73.\newline
%
%%Papers I think are interesting because of features used
%In \cite{Wachsmuth2017Building-an-arg}, a basic framework for argument search engines is created. A common data structure for arguments is developed which makes it possible to transform different argument concepts into it. The prototype\footnote{Available at: http://www.args.me (Last checked: xx.xx.xxxx)} is also capable of retrieving competetive arguments.



\subsection{Domain-Specific Comparative Systems}
The enormous amount of Comparison Portals shows the need for comparisons. Television spots with high production value empathize the popularity of those portals.

Most of those portals are specific to a few domains and a subset of properties, for example, car insurances and their price. Because of that, those systems have some restrictions. Comparisons are only possible between objects of the domains and predefined properties. Source of the data is usually databases. Humans are involved in gathering, entering and processing. 

Comparison Portals solely compare and deliver facts. Because of that, they can only give the advice to choose X over Y based on the facts collected.  An insurance X might be the best in the comparison (e.g., best price), while the internet is full of complaints about lousy service.\newline

Examples of classical Comparative Portals are \emph{Check24, Verivox, Idealo, GoCompare,} and \emph{Compare}\footnote{https://check24.de, https://verivox.de, https://idealo.de, https://gocompare.com, https://compare.com - all last checked: 12.12.2017}, just to name a few.

As an example, Check24. can compare a wide variety of different objects like several insurance types, credit cards, energy providers, internet providers, flights, hotels and car tires. After the user entered some details (based on the object type, see figure \ref{img:check24}), Check24 shows a ranking of different service providers. The user can choose different properties to re-rank the list.
For instance, to compare different DSL providers, the user has to enter her address, how fast the internet should be and if she wants telephone and television as well. She can then select price, speed, and grade (rating) to sort the resulting list.

\begin{figure}[h]
\includegraphics[width=1\textwidth]{images/ds-sys/check24}
\label{img:check24}
\caption{Check24 DSL Provider}
\end{figure}
The other mentioned sites work similarly. They provide more of a ranking than a comparison.\newline


Another interesting type of websites are Question Answering Portals like \emph{Quora} or \emph{GuteFrage}\footnote{https://quora.com, https://gutefrage.net - all last checked: 12.12.2017}. Although comparisons are not their primary goal, a lot of comparative questions are present on those sites.
On Quora, more than 2.380.000 questions have the phrase \enquote{better than} in their title. If \emph{Ruby} and \emph{Python} are added, 10.100 questions remain.\footnote{Checked via Google on 11th of December. Search phrase: \texttt{"better than" site:quora.com} and \texttt{ruby python "better than" site:quora.com}}
Same is true for the German site GuteFrage, though, the numbers are smaller than on Quora.\footnote{334.000 for \texttt{"besser als" site:gutefrage.net} and 78 for \texttt{ruby python "Besser als" site:gutefrage.net}}\newline

More interestingly are systems which can compare any objects on arbitrary properties. Two examples are \emph{Diffen} and \emph{Versus}\footnote{https://diffen.com, https://versus.com - all last checked: 12.12.2017}.

Versus aggregates different freely available data sources like Wikipedia and official statistic reports. For example, the comparison of \enquote{Hamburg vs. Berlin} uses Wikipedia for the number of universities, worldstadiums.com for the availability of sport facilities and the Economist for the Big Mac Index. Presumably, some human processing is involved as the possible comparisons are limited. For instance, a comparison of Hamburg and Darmstadt is not possible as Darmstadt is not available on Versus. Likewise, \enquote{Ruby vs. Python} is not possible, Versus suggests to compare \enquote{Rome vs. Pyongyang} instead. Although Versus shows how many users \enquote{liked} the objects, it does not give a clear statement which one is better. For instance, it is not possible to check automatically whether Hamburg or Berlin is better for a short city trip. The user must search manually all valid properties like the number of museums, theaters, the price of public transport tickets and so on.

Similar to Versus, Diffen aggregates different data sources (see figure \ref{img:diffversus}). All in all, the aggregated information is similar to Versus. The comparison is also tabular. Besides the automatically aggregated data, users can add more information on their own. Diffen describes itself as \enquote{inspired by Wikipedia}\footnote{https://www.diffen.com/difference/Diffen:About - Last checked: 11.12.2017}. Diffen does not enforce any restrictions on the objects of comparison, but it faces the same problem as Versus: objects are missing. A comparison between Darmstadt and Hamburg is likewise not possible: all cells for Darmstadt in the table are just empty.\newline

\begin{figure}[h]
\includegraphics[width=1\textwidth]{images/ds-sys/diffversus}
\label{img:diffversus}
\caption{\enquote{Hamburg vs. Berlin} on Diffen and Versus}
\end{figure}

Neither Versus nor Diffen provides a comprehensible reason why an object is better than another one. They merely aggregate facts and bring them face to face. Despite the aggregation approach of both systems, many meaningful comparisons are not possible or not helpful (\enquote{Hamburg vs. Darmstadt}, \enquote{Java vs. C\#}, \enquote{Dr Pepper vs. Orange Juice}).
Also, the user can not define the properties for the comparison. The sites provide every information available for the objects. For instance, Versus shows 42 properties for \enquote{Hamburg vs. Berlin} and only 35 for \enquote{Hamburg vs. Munich}.
\newline

To summarize, a lot of different comparison portals exist and are widely used. Especially the domain-specific portals do a good job, but inflexibility dearly buys the performance. First, the portals can only compare objects on predefined properties. Second, the data acquisition is not fully automatic. Domain-unspecific systems are good at aggregating information but do not provide a reasonable explanation to prefer X over Y.

Adding information like comments and product reviews can enrich the comparison with reasons and opinions, such as \enquote{Ruby is easier to learn than C} or \enquote{Python is more suitable for scientific applications than Erlang as many libraries exist}.
\chapter{Building a data set for Comparative Argument Mining}
Due to the novelty of Argument Mining (and especially Comparative Argument Mining), the supply of datasets is small. Thus, a new data set had to be created.

This dataset was designed to answer the questions if a given sentence compares two known objects, and if it does, if the first-mentioned object is better or worse than the second one. Those questions will be translated to several classification tasks in the later chapters.

The dataset was created using the crowdsourcing platform CrowdFlower\footnote{https://www.crowdflower.com 23.02.2018}. As described in detail in the following chapters, the annotators were asked to assign one of four (and later three) classes to a sentence in which the objects of interest are highlighted.

The final dataset consists of 7421 sentences, each containing one of 273 object pairs. The sentences were labelled with one of three classes. Each sentence was at least annotated by three different annotators.

\section{Common Crawl Text Corpus}
The sentences for the crowdsourcing campaign were obtained from a CommonCrawl\footnote{https://commoncrawl.org 23.02.2018} dataset. CommonCrawl is a non-profit organisation which crawls the web and releases the crawled data for free use.

The data used in this thesis was already preprocessed\footnote{Download Link} (see \cite{Panchenko:2017aa}). First, it contains only English text. Duplicates and near-duplicates were removed, as well as all HTML tags. The texts were then split into sentences.

The resulting sentences were used to obtain the sentences for the crowdsourcing campaign. To make them manageable, an ElasticSearch index (from now on called "the index") was created. The index contains 3,288,963,864 unique sentences.

To get an idea if there are enough comparative sentences in the index, it was queried for all sentences containing one of the words \enquote{\emph{better}} or \enquote{\emph{worse}},  as those words often indicate a comparison. This query returns 32,946,247 matching sentences. Querying for \enquote{\emph{is better than}} still returns 428,932 sentences.

Those numbers show that there are enough sentences in the index to create a dataset for the given task. Even if only 1\% of the sentences containing \enquote{\emph{is better than}} are truly comparative, there would be 4289 training examples for the machine learning algorithm.


\section{Prestudies}
Before the mainstudy could start, several questions had to be answered.

First, how to extract sentences from the index? Second, how to preprocess those sentences? Third, which labels should be assigned to the sentences? Fourth, how to phrase the guidelines?

Two prestudies were conducted to answer those questions.



\subsection{Sentence Selection}
The sentences for the crowdsourcing campaign should have a high probability of being comparative so that enough positive examples for the machine learning part are present. To ensure this, a list of cue words which indicate comparison was compiled. For the prestudy, those words were \enquote{\emph{better}}, \enquote{\emph{worse}}, \enquote{\emph{inferior}}, \enquote{\emph{superior}}, and \enquote{\emph{because}}. Comparable objects are needed as well. A list of object pairs was selected by hand (see table \ref{tbl:prestudy-objects}). The pairs were selected in a way that they span a wide range of different domains, such as programming languages, countries and pets. The idea behind this is that pets are compared differently than programming languages. In this way, there will be different comparison patterns in the data.

\begin{table}[h]
\centering
\caption{Objects of the Annotation Prestudy}
\label{tbl:prestudy-objects}
\begin{tabular}{@{}llrrr@{}}
\toprule
First Object & Second Object      & \# Sentences                             \\ \midrule
Ruby    & Python    & 100      \\
BMW    & Mercedes    & 100  \\
USA & Europe & 100 \\
Beef & Chicken & 100   \\
Android & iPhone    &   100  \\
Cat & Dog      &     100  \\ 
Football & Baseball   &  100 \\ 
Wine & Beer  & 100  \\
Car & Bicycle & 100 \\
Summer & Winter &  100\\
\bottomrule  
                               
\end{tabular}
\end{table}

However, not all comparisons will contain one of the cue words mentioned above. Two different queries were used to overcome the coverage problem. Sevenhundred-fivtey sentences were obtained using query \ref{lst:es-query-a} (seventy-five for each pair) and 250 using query \ref{lst:es-query-b} (twenty-five for each pair). The second query will also match not-anticipated sentences such as \enquote{\emph{I like X more than Y since Z.}}.



\begin{lstlisting}[label=lst:es-query-a,breaklines=true,postbreak=\mbox{\textcolor{red}{$\hookrightarrow$}\space},caption=Prestudy Sentence Selection Query A]
{ "query":{  "bool":{ "must":[ {
          "query_string":{
            "default_field":"text",
            "query":"(better OR worse OR superior OR inferior) AND \"<OBJECT_A>\" AND \"<OBJECT_B>\""
          }
        } ] } } }
\end{lstlisting}

\begin{lstlisting}[label=lst:es-query-b,breaklines=true,postbreak=\mbox{\textcolor{red}{$\hookrightarrow$}\space},caption=Prestudy Sentence Selection Query B (shortened)]
[...]
          "query_string":{
            "default_field":"text",
            "query":" \"<OBJECT_A>\" AND \"<OBJECT_B>\""
[...]
\end{lstlisting}

Table \ref{tbl:example_sentences} shows some sentences obtained with this method. The objects of interest ar printed in italics.

\begin{table}[h]
\centering
\caption{Extracted Sentences}
\label{tbl:example_sentences}
\begin{tabular}{@{}llr@{}}
\toprule
 Sentence   &  Cue Words Used                      \\ \midrule
 He's the best pet that you can get, Better than a \emph{dog} or \emph{cat}. & Yes \\
\emph{Android} phones have better processing power than \emph{iPhone} & Yes \\
 10 Things \emph{Android} Does Better Than \emph{iPhone} OS & Yes \\
 \emph{Dog} scared of \emph{cat} & No \\
 In fact, many 'supercars' will use \emph{BMW} or \emph{Mercedes} engines. & No \\

\bottomrule  
                               
\end{tabular}
\end{table}



\subsection{Prestudy A}
The first prestudy had two goals. First, it should assess if the sentence selection method returns enough comparative sentences. Second, the design of the study as described below should be checked. On that account, a crowdsourcing campaign with one hundred of the 1000 sentences was started.



For each sentence, the annotators should decide to which class a sentence belongs. The classes are described in table \ref{tbl:prestudyclasses-a}. The classes \texttt{BETTER}, \texttt{WORSE} and \texttt{NO\_COMP} directly refer to the problem stated at the beginning of this chapter. The class \texttt{UNCLEAR} was added to capture all sentences which are somehow comparative but do not fit into the classes \texttt{BETTER} or \texttt{WORSE}.

\begin{figure}[h]
\centering
\includegraphics[width=1\textwidth]{images/prestudy/1_question}
\label{img:1_question}
\caption{Annotator view (Prestudy A)}
\end{figure}

\begin{table}[h]
\centering
\caption{Classes for Prestudy A and B}
\label{tbl:prestudyclasses-a}
\begin{tabular}{@{}ll@{}}
\toprule
Class & Description \\ \midrule
BETTER & The first object in the sentence (object A) is better than the second one (object B)\\
WORSE & The first object is worse \\
UNCLEAR & Neither BETTER nor WORSE fits, but the sentence is comparative\\
NO\_COMP & The sentence is not comparative or the sentence is a question\\
\bottomrule
\end{tabular}
\end{table}

In each sentence, the first object of interest was replaced with \texttt{OBJECT\_A}, while the second one was replaced with \texttt{OBJECT\_B}. Table \ref{tbl:pre_1_res} shows examples of processed sentences. The idea behind this was to enable the annotators to quickly see which objects should be taken into account for assigning a class. For example, in sentence three of table \ref{tbl:pre_1_res}, the annotator might be confused which of the objects are of interest, yet the replacement makes it clear that he should ignore \emph{C} and \emph{VB}. The view of the annotator (for a single sentence) is shown in figure \ref{img:1_question}.

Per batch, each annotator saw five sentences to annotate. He was also able to look into the annotation guidelines anytime he wanted. To filter out low-quality annotators, twelve sentences were selected as test questions. Each participant took a quiz (eight test questions) before the actual annotation process. One of the five sentences of the batch was a test question as well. If the annotator missed more than 30\% of the test questions, he was removed from the task. 

Figure \ref{fig:dist_pre_a} shows the resulting class distribution. The numbers after the class names show the absolute members of that class. As 45\% of the sentences are comparative, the selection procedure works satisfying.

\begin{figure}[h]
\centering
\caption{Class Distribution (Prestudy A)}
\label{fig:dist_pre_a}
\begin{tikzpicture}
\pie [rotate=180, text = legend, color= {cgray, cgreen, cred, cblue}]
    {55/NO\_COMP (55),
    15/BETTER (15),
    8/WORSE (8),
    22/UNCLEAR (22)}
\end{tikzpicture}
\end{figure}

The agreement of the annotators was acceptable. For thirty-seven sentences, all annotators agreed on one classes. Only for five sentences, each annotator assigned a different class. The remaining fifty-eight sentences got two different classes.

\begin{table}[h]
\centering
\caption{Uncertain Sentences (Prestudy A)}
\label{tbl:pre_1_res}
\begin{tabularx}{\textwidth}{lXrrr}
\toprule
\# & Sentence        & Ann. 1  & Ann. 2 & Ann. 3             \\ \midrule
1 & The only reason OBJECT\_A is used over OBJECT\_B, is because of libraries... & WORSE & UNCLEAR & NO\_COMP\\
2 & Agile development is the most popular model at the moment because of architectures like OBJECT\_A on Rails and Django (for OBJECT\_B) & NO\_COMP & NO\_COMP & UNCLEAR\\
3 & Your C# and VB devs can suddenly easily write web apps and your OBJECT\_A and OBJECT\_B devs can too - with the added bonus of much better performance &  NO\_COMP & NO\_COMP & UNCLEAR \\
4 & I'm a huge OBJECT\_A/Django \& OBJECT\_B/Rails fan, but I will never stop using PHP because it is so broadly accepted and supported & NO\_COMP & UNCLEAR & UNCLEAR \\
5 & It's why I mention OBJECT\_A and OBJECT\_B, because they've at least heard of them. & NO\_COMP & NO\_COMP & UNCLEAR \\


\bottomrule                              
\end{tabularx}
\end{table}

Some uncertain sentences are shown in table \ref{tbl:pre_1_res}, which displays the sentence and the decision of each annotator. As one can see in sentence two to five, annotators frequently were not able to distinguish between \texttt{NO\_COMP} and \texttt{UNCLEAR}. This is the case in fifty-one of the fifty-eight cases were two different classes were assigned.\newline

Fourteen out of fifty-five participants took part in an exit survey to rate the task. The overall satisfaction was rated with 3.2 out of 5. While the instructions (4.5), difficulty (4.4) and payment (3.8) got acceptable to good ratings, the test questions (2.9) were critizied. Also, 32 potential annotators failed the quiz. A second prestudy was conducted to adress the discovered problems.







\subsection{Prestudy B}
In the second prestudy, 200 sentences were annotated. Some changes in the task design were made to address the shortcomings of the first study.

Some points were identical to the first study. As the sentence selection process worked fine, the same 1000 base sentences were used in the second prestudy.  Each sentence was annotated by three annotators. They had to pass a quiz of eight test questions and had to keep an accuracy of 70\% on the test questions during the annotation procedure. The classes were the same as well.

The title of the task now contained the information that some computer knowledge is required for this task, as some pairs come from this domain. 
To address the problem with the confusion between \texttt{UNCLEAR} and \texttt{NO\_COMP}, the wording on this classes was changed. The new view of the annotator is displayed in figure \ref{img:2_question}. 
In the first prestudy, some annotators complained that the test questions were not fair. In fact, they contained some special cases so that they did not represent the whole dataset in an appropriate way. In the second prestudy, fifty-one test questions were used, who cover a wider range of examples.

The sentence preprocessing was altered as well. Instead of replacing the object, \mbox{\textbf{{\color[HTML]{9A14B2}:{[}OBJECT\_A{]}}}} or \textbf{{\color[HTML]{6CB219}:{[}OBJECT\_B{]}}} was appended. The colon and square brackets emphasize where the object of interest ends and the suffix begins. The idea behind this was, that the removal of the objects also removes some context from the sentences, which might be needed to classify them correctly. For further highlighting, the objects had a different colour.\newline

\begin{figure}[h]
\centering
\includegraphics[width=1\textwidth]{images/prestudy/2_question}
\label{img:2_question}
\caption{Annotator view (Prestudy B)}
\end{figure}

The resulting class distribution is presentend in figure \ref{fig:dist_pre_b}. 

\begin{figure}[h]
\centering
\caption{Class Distribution (Prestudy B)}
\label{fig:dist_pre_b}
\begin{tikzpicture}
\pie [rotate=180, text = legend, color= {cgray, cgreen, cred, cblue}]
    {48/NO\_COMP (96),
    27/BETTER (54),
    11.5/WORSE (23),
    13.5/UNCLEAR (27)}
\end{tikzpicture}
\end{figure}

As in the first prestudy, nearly half of the sentences are somewhat comparative. For 125 (62.5\%) sentences, all annotators agreed on one class. Four (2.0\%) sentences got three different classes, for the remaining seventy-one (35.5\%) sentences two of the three annotators agreed on one class. The confusion between \texttt{UNCLEAR} and \texttt{NO\_COMP} is still the main problem for the sentences where only two annotators could agree. However, in the second prestudy this confusion only makes up fourty-five out of the seventy-two cases (62.5\% instead of 87.9\% in the first prestudy). Compared to the first prestudy, the amount of sentences where all annotators agreed on one class increased from 37\% to 62.5\%. 

\begin{table}[h]
\centering
\caption{Uncertain Sentences (Prestudy B)}
\label{tbl:pre_1_res}
\begin{tabularx}{\textwidth}{lXrrr}
\toprule
\# & Sentence        & Ann. 1  & Ann. 2 & Ann. 3             \\ \midrule
6 & Google shouldn't have mandated an inferior map app on the iphone:[OBJECT\_A] (as opposed to android:[OBJECT\_B]). & BETTER & WORSE & NO\_COMP \\
7 & (See android:[OBJECT\_A] Dethrones the iphone:[OBJECT\_B] .) & BETTER & NO\_COMP & NO\_COMP \\
8 & android:[OBJECT\_A] didn't out pace the iphone:[OBJECT\_B] this year, it just sold slightly better in America & WORSE & NO\_COMP & WORSE\\
9 & To me it is much better than iphone:[OBJECT\_A] and android:[OBJECT\_B]. & NO\_COMP & UNCLEAR & UNCLEAR \\
10 & ( android:[OBJECT\_A] , Crush , iPad , iphone:[OBJECT\_B] )& NO\_COMP & BETTER & NO\_COMP \\

\bottomrule                              
\end{tabularx}
\end{table}


From 125 candidate annotators, thirty-five failed the initial quiz. Twelve annotators were removed during the annotation process as they answered too many test questions wrong.
Twenty-two annotators took the exit survey. The overall satisfaction increased to 3.7 out of 5. The test question fairness was now rated with 3.7 out of 5 instead of 2.9. The rating for the payment slightly increased to 3.9, yet the payment was not changed. However, the rating for the instructions decreased to 3.9 and for the difficulty to 3.5.
The change in numbers is explained by the increased amount of sentences, which introduce new cases which are not directly reflected in the annotation guidelines. However, as only a small fraction of the annotators took the exit survey in both prestudies, the results can only be used as one signal. The annotation results are another, more important signal, and they are convincing.

\subsection{Validation of results}

Due to an error in the creation of the crowdsourcing task, the sentences where not shuffled. This means that the first one-hundred sentences of the second prestudy are the same as the one-hundred sentences of the first prestudy. Another problem is the bias: except for X sentences, all other sentences contained the pairs X and Y. However, since the goal of the prestudy was mainly to assess the sentence selection method and the guidelines, this does not invalidate the results. This problems were removed in the main study.
%\subsection{Task}
%
%\textbf{ALT ALT ALT ALT ALT}
%Using the method described above, 1050 sentences were obtained for the prestudy. The annotators were asked to assigne one of the following classes to the sentences. Each sentence was annotated by three annotators.
%
%The annotators where asked to assign one of the four classes (see table \ref{tbl:prestudyclasses}) to each sentence.
%
%
%
%
%
%In a first step, 100 sentences were annotated. To ensure the quality, twelve additional sentences were setup as test sentences. If one annotator failed three test sentences, he was removed from the task.
%
%The sentences were preprocessed: the first object was replaced by OBJECT\_A, the second by OBJECT\_B. Examples are shown in table \ref{tbl:pre1s}. The removal was done so that the annotators can concentrate on the comparative structure of the sentence and are not biased by the objects.
%
%
%% Beispielsätze PreStudy 1. Teil
%
%
%
%This test step delivered valuable insights. First, the amount of test sentences was to small. Users might see the same test sentence twice. Second, the phrasing of the annotation guidelines was to confusing, especially the distinction between NO\_COMP and UNCLEAR as well as their class names.
%Third, the complete removal of the original objects is suspected to partly obscure the sense of the sentences.\newline
%
%In a second step, 200 new sentences were annotated, again with three annotations per sentences. This time, 51 test questions were used, so that it is less likely that annotators will see the same question twice. Furthermore, the preprocessing was changed. Instead of removing the original objects, :[OBJECT\_A] was appended to the first object, :[OBJECT\_B] to the second object. Also, each object was highlighted in a different color. Example sentences are shown in table \ref{tbl:pre2s}. In this way, the annotators could quickly see the objects of interest while the sense of the sentence remains intact.
%% Beispielsätze PreStudy 2. Teil
%\begin{table}[h]
%\centering
%\caption{Sentences for the second step}
%\label{tbl:pre2s}
%\begin{tabular}{{p{12cm}p{3cm}}}
%\toprule
%Sentence                                                                                                           & Expected Class \\ \midrule
%I'd go with \textbf{{\color[HTML]{9A14B2} python:{[}OBJECT\_A{]}}} or \textbf{{\color[HTML]{6CB219}ruby:{[}OBJECT\_B{]}}}.                                 & NO\_COMP       \\
%I prefer \textbf{{\color[HTML]{9A14B2}ruby:{[}OBJECT\_A{]}}} over \textbf{{\color[HTML]{6CB219}python:{[}OBJECT\_B{]}}} on windows.                                              & BETTER         \\
%I've tried \textbf{{\color[HTML]{9A14B2}python:{[}OBJECT\_A{]}}}, and can see why people like it, but \textbf{{\color[HTML]{6CB219}ruby:{[}OBJECT\_B{]}}} suits my style better. & WORSE          \\
%i think this car is a far better deal than the \textbf{{\color[HTML]{9A14B2}bmw{:[OBJECT\_A]}}} 5 series or \textbf{{\color[HTML]{6CB219}mercedes:[OBJECT\_B]}} 320e.                                                                                                                &        UNCLEAR        \\ \bottomrule
%\end{tabular}
%\end{table}
%
%\label{sec:annotation-guidelines}
%\subsection{Results}
%Each sentence was annotated by three annotators. Figure \ref{pre:dist} shows the class distribution.
%
%\begin{figure}[h]
%\centering
%\caption{Class Distribution in the prestudy}
%\label{pre:dist}
%\begin{tikzpicture}
%\pie [rotate=180, text = legend, color= {cgray, cgreen, cred, cblue}]
%    {59.76/NO\_COMP (150),
%    23.11/BETTER (58),
%    9.16/WORSE (23),
%    7.97/UNCLEAR (20)}
%\end{tikzpicture}
%\end{figure}
%
%
%%\includegraphics[scale=0.6]{images/prestudy/label_distribution.pdf}
%
%
%Crowdflower has a trust value for each annotator. This trust value and the number of votes per class gives a value of confidence for each label.\footnote{How the confidence is calculated in detail can be found at https://success.crowdflower.com/hc/en-us/articles/201855939-How-to-Calculate-a-Confidence-Score (Last checked: 19.12.2017)}
%
%
%As presented in figure \ref{pre:conf}, a majority (151) of the labelings has a confidence greater or equal to 0.9, and 15 sentences a confidence below 0.6; the mean is 0.86. Detailed numbers on the confidence are shown in table \ref{pre:conf-table}
%
%\begin{figure}
%\centering
%\caption{Confidence histogram}
%\label{pre:conf}
%\includegraphics[scale=0.6]{images/prestudy/confidence.pdf}
%\end{figure}
%
%
%\begin{figure}[h]
%\centering
%\caption{Confidence}
%\begin{tabular}{@{}ll@{}}
%\toprule
%Type & Value  \\ \midrule
%Average Confidence & 0.86 \\
%Standard Derivation & 0.17 \\
%Lowest Confidence & 0.35\\
%Highest Confidence & 1.00\\
%25th percentile average & 0.67\\
%50th percentile average & 1.00\\
%\bottomrule
%\label{pre:conf-table}
%\end{tabular}
%\end{figure}
%
%
%
%
%
%The most difficult sentence is with a confidence of 0.35 for the class \emph{WORSE} was
%\begin{quote}
%Google shouldn't have mandated an inferior map app on the iphone:[OBJECT\_A] (as opposed to android:[OBJECT\_B]).
%\end{quote}
%
%It was labelled as \emph{BETTER} (trust: 0.72), \emph{WORSE} (trust: 0.85) and \emph{NO\_COMP} (trust: 0.82). The class \emph{WRONG} is correct here, as the object \enquote{iphone} is inferior to \enquote{android} on the aspect of \enquote{map app}.
%
%The following sentence was assigned to \emph{BETTER} (0.37 confidence), although it should belong to \emph{UNCLEAR}.
%\begin{quote}
%Not to mention that the iphone:[OBJECT\_A] and android:[OBJECT\_B] phones deliver a far superior user experience overall
%\end{quote}
%However, the annotator for \emph{UNCLEAR} only had 0.87 trust, while the one for \emph{BETTER} had 1 (third one was \emph{NO\_COMP} with 0.82 trust).\newline
%
%All things considered, the result of the prestudy is satisfactory. The annotators agreed in the majority of decisions. 


\newpage
\section{Main Study}
\label{sec:mainstudy}
\subsection{Task Description}
\subsection{Data Generation}
Three domains were fixed for the sentences of the main study. The domains were chosen in a way that a majority of people can decide whether a sentence contains a comparison or not.

The most specific domain was "Computer Science Concepts". It contains objects like programming languages, database products and technology standards such as Bluetooth and Ethernet.  Many computer science concepts can be compared objectively, for instance, one can compare Bluetooth and Ethernet on their transmission speed. Some basic knowledge of computer science was needed to label sentences correctly. For example, to compare Eclipse and NetBeans, the annotator must know what an Integrated Development Environment (IDE) is and that both objects are Java IDEs.  The need of the knowledge was communicated to the prospective annotators. The objects for this domain were manually extracted from "List of ..." articles from Wikipedia.

The second, broader domain was "Brands". It contains objects from of different types (car brands, electronics brands, and food). As brands are present in everyday life of people, it is expected that anyone can label the majority of sentences containing well known brands such as Coca-Cola or Mercedes. As with computer science, the objects for this domain were extracted from "List of ..." articles from Wikipedia.

The last domain is not restricted to any topic. For each one of 25 randomly selected seed words, ten similar words were extracted using JoBimText, a software package for distributional semantics. The seed words were created using https://randomlists.com\footnote{Last checked: 25.01.2018}. Listing \ref{lst:jbtres} shows the result\footnote{http://ltmaggie.informatik.uni-hamburg.de/jobimviz/ws/api/stanford/jo/similar/harvard\%23NP?numberOfEntries=10&format=json Last checked: 25.01.2018; Some uninteresting fields were removed for brevity} for the seed word \emph{harvard}.

\begin{lstlisting}[language=json,label=lst:jbtres,caption=Similar words to "Harvard"]
{
   "results":[
      { "score":688.0, "key":"harvard#NP" },
      { "score":245.0, "key":"yale#NP" },
      { "score":163.0, "key":"princeton#NP" },
      { "score":152.0, "key":"mit#NP" },
      { "score":143.0, "key":"stanford#NP" },
      { "score":133.0, "key":"university#NP"},
      { "score":132.0, "key":"tufts#NP" },
      { "score":130.0,"key":"cornell#NP"},
      { "score":127.0, "key":"nyu#NP" },
      { "score":113.0, "key":"university#NN" }
   ]
}
\end{lstlisting}
This method covers a wide are of possible comparison patterns.\newline

Especially for brands and computer science, the object lists are long (4493 brands and 1339 for computer science).The frequency of each object was checked using a frequency dictionary to reduce the number of possible pairs. All objects with a frequency of zero and ambiguous objects were removed from the list. For instance, the objects "RAID" (a hardware concept) and "Unity" (a game engine) were removed from the computer science list as they are also regularly used nouns.

The remaining objects were combined to pairs. For each type, all possible combinations were created. For brands and computer science, the type is the source list. For the unrestricted domain, the seed word was used. This procedure guarantees that only meaningful pairs are created.
The ElasticSearch Index was then queried for entries containing both objects of each pair. For 90\% of the queries, the marker terms where added to the query. This was done to check whether there is a chance that those two objects were compared. All pairs were the query yielded at least 100 sentences were kept. Those pairs are frequent enough and have a high chance of generating comparative sentences.

From the sentences of those pairs, 2500 for each category were randomly sampled as candidates for the crowdsourcing campaign. 250 sentences were manually labelled to check if there are enough comparative sentences. Those labels were discarded for the crowdsourcing campaign.
The label distribution of the 250 sentences is presented in the figures FIGURE NUMBERS.

\begin{figure}[h]
    \centering
    \begin{minipage}{0.49\textwidth}
        \centering
       \begin{tikzpicture}
\pie [rotate=180,radius=2,color= {cgray, cgreen, cred, cblue}]
    {68.9/NONE,
    15.2/BETTER,
    5.7/WORSE,
    10.2/OTHER }
    \label{pre:brands}
\end{tikzpicture}        \caption{Brands}
    \end{minipage}\hfill
    \begin{minipage}{0.49\textwidth}
        \centering
      \begin{tikzpicture}
\pie [rotate=180,radius=2, color= {cgray, cgreen, cred, cblue}]
    {53.00/NONE,
    19.70/BETTER,
    11.60/WORSE,
    15.70/OTHER }
\end{tikzpicture}
        \caption{Computer Science}
    \end{minipage}
    \end{figure}
    
    \begin{figure}[h]
    \centering
    \begin{minipage}{0.49\textwidth}
        \centering
       \begin{tikzpicture}
\pie [rotate=180,radius=2,color= {cgray, cgreen, cred, cblue}]
    {65.50/NONE,
    16.10/BETTER,
    7.20/WORSE,
    11.20/OTHER }
\end{tikzpicture}        \caption{Unrestricted}
    \end{minipage}\hfill
    \end{figure}
    
In all samples, at least 30\% of the sentences are comparative. This number shows that the sampling method is sufficient to sample sentences for the crowdsourcing campaign.

\chapter{Classification of Comparative Sentences}
\section{Experiments}
The data collected from the crowdsourcing task was used as training data for classification problems. In the first problem, the machine learning algorithms were trained so that they were able to assign one of the three original classes (see section \ref{sec:designchanges}) to the data. The second problem is a simplification of the first one as it is designed as a binary classification problem. The classes \texttt{BETTER} and \texttt{WORSE} were merged into the class \texttt{ARG}.


The data was split into a training set (5937 examples; 4270 \texttt{NONE}, 1158 \texttt{BETTER} and 509 \texttt{WORSE}) and a held-out set. The held-out set stayed untouched until the final evaluation presented in section \ref{sec:final}. During the development, the experiments were evaluated using stratified k-fold cross-validation where k equals five. The evaluation was done with the scikit-learn implementation. As the evalutation metric, the weighted mean\footnote{weighted by number of examples per class} of the F1 scores of all classes is used. 

\section{Choice of Algorithms}

To find the best performing classification algorithms, eleven (see table \ref{tbl:algo}) were selected and compared. For all algorithms under test, the unigrams of the whole sentence were used as binary features (the implementation is described in section \ref{sec:ngrams}). Unigrams are a simple yet efficient feature for text classification (\cite{cavnar1994n}) which makes them suitable as a baseline feature for comparisons. Stratified k-fold with k equals five was used to assess the quality of the algorithm. The unweighted average of the five F1 scores was used to make the different algorithms more comparable.

\begin{table}[h]
\centering
\label{tbl:algo}
\caption{All evaluated classification algorithms. The F1 score shows the classification performance with the unigram feature.}
\begin{tabularx}{\textwidth}{XlX}
\toprule
Algorithm & F1 Score & Used in\\ \midrule

XGBoost & 0.76 & -\\ 

Logistic Regression & 0.75 &  \cite{Dusmanu2017Argument-Mining,Daxenberger2017What-is-the-EssAker2017What-works-and-,Lippi2016Argumentation-M}\\ 

AdaBoost & 0.74 &  \cite{Aker2017What-works-and-}\\  

Linear SVC & 0.73 & \cite{Aker2017What-works-and-}\\  

Decision Tree & 0.72 &  \cite{Stab2014Identifying-Arg,Lippi2016Argumentation-M}\\ 

Stochastic Gradient Descent & 0.72 & -\\ 


Random Forest & 0.72 &  \cite{Dusmanu2017Argument-Mining,Stab2014Identifying-Arg,Eckle-Kohler2015On-the-Role-of-,Aker2017What-works-and-,Lippi2016Argumentation-M}\\  

Extra Trees Classifier & 0.72 & - \\

K Neighbors & 0.72 &  \cite{Aker2017What-works-and-}\\ 


Support Vector Machine (non-linear kernel) & 0.60 & \cite{Stab2014Identifying-Arg,Eckle-Kohler2015On-the-Role-of-,Park:2012:ICC:2391171.2391173,Lippi2016Argumentation-M,Habernal2016Argumentation-M}\\ 


Naive Bayes & 0.60 & \cite{Stab2014Identifying-Arg,Eckle-Kohler2015On-the-Role-of-,Aker2017What-works-and-,Park:2012:ICC:2391171.2391173,Lippi2016Argumentation-M}\\  


 \bottomrule 
\end{tabularx}

\end{table}


For all algorithms except XGBoost, the implementation available from scikit-learn (\cite{scikit-learn} ) was used. Tree-based methods and simple linear models work good. The \emph{Support Vector Machine} without a linear kernel was not able to classify the examples at all as it assigns \texttt{NONE} to all training examples. This is true for all non-linear kernel functions available in scikit-learn (RBF, polynomial and Sigmoid).

Naive Bayes classifies examples correctly into all classes, yet the performance is still same as the baseline shown in section \ref{sec:baseline}.

As XGBoost and Logistic Regression already work in a pleasing way, no further investigations on the performance of the \emph{Support Vector Machine} and \emph{Naive Bayes} were done. In the following sections, all experiments were done using XGBoost.

A set of hyper-parameters for XGBoost were tested using exhaustive grid search and randomized search. However, no significant increase in the F1 score could be achieved.








\section{Features}
The following section briefly describes each feature used for both classification tasks. 

Two observations are true for all features. First, the F1 score increases by at least five points if only the part of the sentence between both objects of interest is used (from now on called \enquote{\emph{middle part}}).

 Second, the objects do not make any significant difference. All features were tested with four versions of the sentence (eventually with the middle part only). 
\begin{enumerate}
\item the sentence stayed unchanged
\item the objects of interest removed completely from the sentence
\item both objects replaced with \emph{OBJECT}
\item the first object replaced with \emph{OBJECT\_A}, the second object replaced with \emph{OBJECT\_B}
\end{enumerate}
Most of the time, the F1 score for the different versions differ - if at all - in the third decimal place. In exceptional cases, using the fourth option increases the F1 score by X points. If not stated differently, all features used the unchanged middle part of the original sentence.


\subsection{Bag-Of-Words Feature}
Some variants of the BOW model present in section \ref{sec:bow_model} have been used. Binary feature vectors\footnote{the n-grams were computed using \emph{textacy} (https://github.com/chartbeat-labs/textacy)} and their tf-idf weighted\footnote{relying on the scikit-learn implementation} counterparts were tested for uni-, bi- and trigrams. No restrictions were imposed on the n-grams.

Uni- and bigram models bet both baselines.

The idea behind using trigram models despite the short text length was to capture trigrams like \enquote{\emph{is better than}} or \enquote{\emph{is worse because}} which would indicate a comparison. However, the performance of trigram models was not satisfying.
\label{sec:ngrams}

\subsection{Part of Speech}
Several boolean features were used to model the apperance of parts-of-speech in the sentence. The part-of-speech tagging was done with \emph{spaCy}. The feature capturing the apperance of the part-of-speeches \emph{comparative adverb (RBR)} and \emph{comparative adjective (JJR)}\footnote{https://www.ling.upenn.edu/courses/Fall\_2003/ling001/penn\_treebank\_pos.html} had an equal performance to the unigram model. Other part-of-speech tags did not get a result above the baseline (for instance \emph{superlative adjective (RBS)} However, using this feature in isolation, no training example is assigned to the class ?

\subsection{Mean Word Embeddings}

\subsection{Sentence Embeddings}
\subsection{Dependency Features}
\subsection{Other Features}
A couple of features did not contribute to the classification at all, be it alone or in combination with other features. This includes the length of the sentence or its parts, statistics over punctuation or named entities.
% n-grams
% sentence embeddings
% POS JJR

\section{Classification with three classes}
\subsection{Baseline}
\label{sec:baseline}
As described in section \ref{sec:argmine}, there is no task which is similar enough to this one which could be used as a baseline. Thus, two baselines using the obtained data were created. The first baseline, shown in table \ref{foo}, assigns all sentences to the class \texttt{NONE}.


\begin{table}[!htb]
    \begin{minipage}{.5\linewidth}
      \caption{Random baseline for the binary classification task.}
      \label{foo}
      \centering
\begin{tabular}{@{}lrrrr@{}}
\toprule
 	&	 precision &	 recall &	 f1 score  \\ \midrule 
ARG	&	 0.00 \scriptsize{(0.00)} &	 0.00 \scriptsize{(0.00)} &	 0.00 \scriptsize{(0.00)}  \\ 
NONE	&	 0.80 \scriptsize{(0.03)} &	 1.00 \scriptsize{(0.00)} &	 0.89 \scriptsize{(0.02)}  \\ 
average	&	 0.64 \scriptsize{(0.04)} &	 0.80 \scriptsize{(0.03)} &	 0.71 \scriptsize{(0.04)}  \\ 
\bottomrule
\end{tabular}
    \end{minipage}%
    \begin{minipage}{.5\linewidth}
      \centering
        \caption{Majority class baseline for the binar classification task.}
\begin{tabular}{@{}lrrrr@{}}
\toprule
 	&	 precision &	 recall &	 f1 score  \\ \midrule 
ARG	&	 0.00 \scriptsize{(0.00)} &	 0.00 \scriptsize{(0.00)} &	 0.00 \scriptsize{(0.00)}  \\ 
NONE	&	 0.80 \scriptsize{(0.03)} &	 1.00 \scriptsize{(0.00)} &	 0.89 \scriptsize{(0.02)}  \\ 
average	&	 0.64 \scriptsize{(0.04)} &	 0.80 \scriptsize{(0.03)} &	 0.71 \scriptsize{(0.04)}  \\ 
\bottomrule
\end{tabular}
    \end{minipage} 
\end{table}



The second baseline is created by assigning classes to the data at random, respecting the distribution of classes in the original data. The results are shown in table \ref{tbl:a}.

For both baselines, the scikit-learn's \texttt{DummyClassifer} was used.



\subsection{Results}

\begin{table}[!htb]
    \caption{Global caption}
    \begin{minipage}{.52\linewidth}
      \caption{Bla blubb}
      \centering
 \begin{tabular}{@{}lrrrr@{}}
\toprule
 	&	 precision &	 recall &	 f1 score  \\ \midrule 
\texttt{BETTER}	&	 0.63 \scriptsize{(0.06)} &	 0.34 \scriptsize{(0.03)} & 0.45 \scriptsize{(0.03)}  \\ 
\texttt{WORSE}	&	 0.33 \scriptsize{(0.16)} &	 0.01 \scriptsize{(0.01)} & 0.02 \scriptsize{(0.01)}  \\ 
\texttt{NONE}	&	 0.79 \scriptsize{(0.01)} &	 0.97 \scriptsize{(0.01)} & 0.87 \scriptsize{(0.00)}  \\ \midrule
average	&	 0.72 \scriptsize{(0.02)} &	 0.77 \scriptsize{(0.01)} & 0.72 \scriptsize{(0.01)}  \\ 
\bottomrule
\end{tabular}
    \end{minipage}%
    \begin{minipage}{.5\linewidth}
      \centering
        \caption{Blubb}
 \begin{tabular}{@{}lrrrr@{}}
\toprule
 	&	 precision &	 recall &	 f1 score  \\ \midrule 
\texttt{BETTER}	&	 0.63 \scriptsize{(0.06)} &	 0.34 \scriptsize{(0.03)} & 0.45 \scriptsize{(0.03)}  \\ 
\texttt{WORSE}	&	 0.33 \scriptsize{(0.16)} &	 0.01 \scriptsize{(0.01)} & 0.02 \scriptsize{(0.01)}  \\ 
\texttt{NONE}	&	 0.79 \scriptsize{(0.01)} &	 0.97 \scriptsize{(0.01)} & 0.87 \scriptsize{(0.00)}  \\ \midrule
average	&	 0.72 \scriptsize{(0.02)} &	 0.77 \scriptsize{(0.01)} & 0.72 \scriptsize{(0.01)}  \\ 
\bottomrule
\end{tabular}
    \end{minipage} 
\end{table}

%==== result table     


\section{Binary classification}
\subsection{Baseline}

Table \ref{tbl:bin_maj}; Table \ref{tbl:bin_strat}

 \begin{table}[h]
                \centering
\caption{Random baseline for the binary classification task. The classes are assigned at random, but the distribution in the original data set is kept.}
\label{tbl:bin_strat}
 \begin{tabular}{@{}lccccccccc@{}}
              \toprule
               & \multicolumn{3}{c}{Worst} & \multicolumn{3}{c}{Average} & \multicolumn{3}{c}{Best}  \\ \midrule
               & Precision  & Recall & F1   & Precision  & Recall  & F1    & Precision & Recall & F1   \\ \toprule
\texttt{ARG}	 & 0.25	 & 0.26	 & 0.25	 &0.29	 & 0.29	 & 0.29	 &0.30	 & 0.30	 & 0.30	 \\ 
\texttt{NONE}	 & 0.71	 & 0.70	 & 0.71	 &0.72	 & 0.72	 & 0.72	 &0.73	 & 0.72	 & 0.72	 \\ \midrule 
average	 & 0.58	 & 0.58	 & 0.58	 &0.60	 & 0.60	 & 0.60	 &0.61	 & 0.60	 & 0.60	 \\ \bottomrule

    \end{tabular}
\end{table}

 \begin{table}[h]
                \centering
\caption{Majority class baseline for the binar classification task.}
\label{tbl:bin_maj}
 \begin{tabular}{@{}lccccccccc@{}}
              \toprule
               & \multicolumn{3}{c}{Worst} & \multicolumn{3}{c}{Average} & \multicolumn{3}{c}{Best}  \\ \midrule
               & Precision  & Recall & F1   & Precision  & Recall  & F1    & Precision & Recall & F1   \\ \toprule
\texttt{ARG}	 & 0.00	 & 0.00	 & 0.00	 &0.00	 & 0.00	 & 0.00	 &0.00	 & 0.00	 & 0.00	 \\ 
\texttt{NONE}	 & 0.72	 & 1.00	 & 0.84	 &0.72	 & 1.00	 & 0.84	 &0.72	 & 1.00	 & 0.84	 \\ \midrule 
average	 & 0.52	 & 0.72	 & 0.60	 &0.52	 & 0.72	 & 0.60	 &0.52	 & 0.72	 & 0.60	 \\ \bottomrule

    \end{tabular}
\end{table}

\subsection{Results}

\section{Final results}
\label{sec:final}

\section{Discussion}
\chapter{Conclusion and future work}
%hypnet features might work \texttt{BETTER} with more examples -> sample from the %index
The thesis dealt with the problem of Comparative Argument Mining. 

The first part discussed the creation of a labelled data set which contains a wide range of comparative sentences.

The second part discussed how to create a machine learning system which is able to classify the sentences in the created data set. \emph{Gradient Boosted Decision Trees} turned out to be the best classifier for this task. Various simple (like bag-of-words) and complex features (like sentence embeddings) achieved f1 scores at least ten points over the baseline. As presented in section \ref{sec:3_results}, the f1 score was greatly increased by some preprocessing steps. It turned out that the words between the two compared objects are most important. Features calculated with only these words outperformed all features calculated with the whole sentence.

The simplification from a three-class problem to a binary problem (by merging the comparative classes \texttt{BETTER} and \texttt{WORSE} into one class \texttt{ARG}) increased the performance.

The final evaluation on unseen data showed that most features generalise well. All in all, the classification works satisfactory.
\newline


Some aspects were not covered in this thesis. As described in section \ref{sec:mainstudy}, the data set was created on the sentence level. Because of this, no context information is available for the classification. However, the context can hold important information. For instance, the presented system does not work with a sentence like \emph{\enquote{This is better than Java.}} because the second object is missing. The preceding sentence might contain the object which is referenced by \emph{\enquote{This}}. 

Section \ref{sec:3_results} showed that the features based on LexNet yield acceptable results. It is expected that the results would increase if more data is available to create the path embeddings. In \cite{DBLP:conf/acl/ShwartzGD16} and \cite{DBLP:journals/corr/ShwartzD16}, the systems were trained on a Wikipedia corpus, which is magnitudes larger than the 7199 sentences from the corpus created in this thesis. One (costly) approach for future work is to annotate more data. Another approach could sample new sentences from the index, by just using patterns like \emph{\enquote{is better than}} or \emph{\enquote{is worse than}}. The quality would not be so good as with manually labelled data, but ths might be compensated by the neural network if it is trained long enough.

The results in \ref{sec:final} show that several features generalise well. The f1 score for unseen data is comparable to the scores during the development phase. Yet, the system was not tested in a real world application. This could be a comparison search engine which takes two objects as the input and returns all comparisons. In a next step, the search engine could inspect the retrieved comparions and extract compared properties and the like.

\appendix

	\counterwithin{table}{section}
	\chapter{Detailed Classification Results}
\section{Feature Experiments}
	\setcounter{section}{1}
	The following shows the classification result for each feature. Each feature was tested with five stratified folds. The result is presented as the average out of five folds with standard derivation. The class \texttt{ARG} is the union of \texttt{BETTER} and \texttt{WORSE}.
	

	
	\begin{table}[h] 
		\centering 
		\caption{Bag-Of-Words feature (three-class scenario). The presence of all unigrams in the corpus are represented as binary features.} 
		\label{  }
		\begin{tabular}{@{}lrrrr@{}}
			\toprule
			        & precision                & recall                   & f1 score                 \\ \midrule 
			\texttt{BETTER}  & 0.80 \scriptsize{(0.02)} & 0.72 \scriptsize{(0.02)} & 0.75 \scriptsize{(0.01)} \\ 
			\texttt{WORSE}   & 0.62 \scriptsize{(0.06)} & 0.39 \scriptsize{(0.05)} & 0.48 \scriptsize{(0.05)} \\ 
			\texttt{NONE}    & 0.89 \scriptsize{(0.01)} & 0.95 \scriptsize{(0.01)} & 0.92 \scriptsize{(0.00)} \\ 
			average & 0.85 \scriptsize{(0.01)} & 0.86 \scriptsize{(0.00)} & 0.85 \scriptsize{(0.01)} \\ 
			\bottomrule
		\end{tabular}
	\end{table}
	
	
	\begin{table}[h] 
		\centering 
		\caption{Bag-Of-Words feature (binary scenario). The presence of all unigrams in the corpus are represented as binary features.} 
		\label{  }
		\begin{tabular}{@{}lrrrr@{}}
			\toprule
			        & precision                & recall                   & f1 score                 \\ \midrule 
			\texttt{ARG}     & 0.80 \scriptsize{(0.02)} & 0.74 \scriptsize{(0.01)} & 0.77 \scriptsize{(0.01)} \\ 
			\texttt{NONE}    & 0.91 \scriptsize{(0.00)} & 0.93 \scriptsize{(0.01)} & 0.92 \scriptsize{(0.00)} \\ 
			average & 0.88 \scriptsize{(0.00)} & 0.88 \scriptsize{(0.00)} & 0.88 \scriptsize{(0.00)} \\ 
			\bottomrule
		\end{tabular}
	\end{table}
	
	
	
	\begin{table}[h] 
		\centering 
		\caption{InferSent (sentence embeddings) feature (three-class scenario).} 
		\label{  }
		\begin{tabular}{@{}lrrrr@{}}
			\toprule
			        & precision                & recall                   & f1 score                 \\ \midrule 
			\texttt{BETTER}  & 0.79 \scriptsize{(0.02)} & 0.73 \scriptsize{(0.03)} & 0.76 \scriptsize{(0.02)} \\ 
			\texttt{WORSE}   & 0.53 \scriptsize{(0.07)} & 0.34 \scriptsize{(0.02)} & 0.41 \scriptsize{(0.03)} \\ 
			\texttt{NONE}    & 0.90 \scriptsize{(0.01)} & 0.95 \scriptsize{(0.00)} & 0.92 \scriptsize{(0.00)} \\ 
			average & 0.85 \scriptsize{(0.01)} & 0.86 \scriptsize{(0.01)} & 0.85 \scriptsize{(0.01)} \\ 
			\bottomrule
		\end{tabular}
	\end{table}
	
	\begin{table}[h] 
		\centering 
		\caption{InferSent (sentence embeddings) feature (binary scenario).} 
		\label{  }
		\begin{tabular}{@{}lrrrr@{}}
			\toprule
			        & precision                & recall                   & f1 score                 \\ \midrule 
			\texttt{ARG}     & 0.79 \scriptsize{(0.01)} & 0.78 \scriptsize{(0.02)} & 0.78 \scriptsize{(0.01)} \\ 
			\texttt{NONE}    & 0.92 \scriptsize{(0.01)} & 0.92 \scriptsize{(0.01)} & 0.92 \scriptsize{(0.00)} \\ 
			average & 0.88 \scriptsize{(0.01)} & 0.88 \scriptsize{(0.00)} & 0.88 \scriptsize{(0.00)} \\ 
			\bottomrule
		\end{tabular}
	\end{table}
	
	
	\begin{table}[h] 
		\centering 
		\caption{Mean Word Embeddings (three-class scenario). All GloVe word vectors of a sentence were summed up and divided by the number of words in the sentence.} 
		\label{  }
		\begin{tabular}{@{}lrrrr@{}}
			\toprule
			        & precision                & recall                   & f1 score                 \\ \midrule 
			\texttt{BETTER}  & 0.71 \scriptsize{(0.01)} & 0.73 \scriptsize{(0.04)} & 0.72 \scriptsize{(0.02)} \\ 
			\texttt{WORSE}   & 0.47 \scriptsize{(0.05)} & 0.16 \scriptsize{(0.02)} & 0.24 \scriptsize{(0.03)} \\ 
			\texttt{NONE}    & 0.88 \scriptsize{(0.01)} & 0.94 \scriptsize{(0.00)} & 0.91 \scriptsize{(0.00)} \\ 
			average & 0.82 \scriptsize{(0.01)} & 0.84 \scriptsize{(0.01)} & 0.82 \scriptsize{(0.01)} \\ 
			\bottomrule
		\end{tabular}
	\end{table}
	
	\begin{table}[h] 
		\centering 
		\caption{Mean Word Embeddings (binary class scenario). All GloVe word vectors of a sentence were summed up and divided by the number of words in the sentence.} 
		\label{  }
		\begin{tabular}{@{}lrrrr@{}}
			\toprule
			        & precision                & recall                   & f1 score                 \\ \midrule 
			\texttt{ARG}     & 0.78 \scriptsize{(0.01)} & 0.74 \scriptsize{(0.01)} & 0.76 \scriptsize{(0.01)} \\ 
			\texttt{NONE}    & 0.91 \scriptsize{(0.00)} & 0.92 \scriptsize{(0.00)} & 0.91 \scriptsize{(0.00)} \\ 
			average & 0.87 \scriptsize{(0.01)} & 0.88 \scriptsize{(0.01)} & 0.87 \scriptsize{(0.01)} \\ 
			\bottomrule
		\end{tabular}
	\end{table}
	

	\begin{table}[h] 
		\centering 
		\caption{N-gram POS feature (three-class scenario). The presence of the 500 most frequent part-of-speech bi-, tri- and four-grams were represented as binary features.} 
		\label{  }
		\begin{tabular}{@{}lrrrr@{}}
			\toprule
			        & precision                & recall                   & f1 score                 \\ \midrule 
			\texttt{BETTER}  & 0.57 \scriptsize{(0.01)} & 0.59 \scriptsize{(0.03)} & 0.58 \scriptsize{(0.02)} \\ 
			\texttt{WORSE}   & 0.26 \scriptsize{(0.02)} & 0.11 \scriptsize{(0.02)} & 0.15 \scriptsize{(0.02)} \\ 
			\texttt{NONE}    & 0.87 \scriptsize{(0.01)} & 0.92 \scriptsize{(0.02)} & 0.89 \scriptsize{(0.01)} \\ 
			average & 0.76 \scriptsize{(0.01)} & 0.79 \scriptsize{(0.01)} & 0.77 \scriptsize{(0.01)} \\ 
			\bottomrule
		\end{tabular}
	\end{table}
	
	\begin{table}[h] 
		\centering 
		\caption{N-gram POS feature (three-class scenario). The presence of the 500 most frequent part-of-speech bi-, tri- and four-grams were represented as binary features.} 
		\label{  }
		\begin{tabular}{@{}lrrrr@{}}
			\toprule
			        & precision                & recall                   & f1 score                 \\ \midrule 
			\texttt{ARG}     & 0.70 \scriptsize{(0.02)} & 0.69 \scriptsize{(0.01)} & 0.70 \scriptsize{(0.01)} \\ 
			\texttt{NONE}    & 0.89 \scriptsize{(0.00)} & 0.89 \scriptsize{(0.01)} & 0.89 \scriptsize{(0.00)} \\ 
			average & 0.84 \scriptsize{(0.00)} & 0.84 \scriptsize{(0.01)} & 0.84 \scriptsize{(0.00)} \\ 
			\bottomrule
		\end{tabular}
	\end{table}
	

	

	\begin{table}[h] 
		\centering 
		\caption{Binary feature which represents the presence of a comparative adjective in the sentence (three-class scenario).} 
		\label{ }
		\begin{tabular}{@{}lrrrr@{}}
			\toprule
			        & precision                & recall                   & f1 score                 \\ \midrule 
			\texttt{BETTER}  & 0.56 \scriptsize{(0.01)} & 0.62 \scriptsize{(0.04)} & 0.59 \scriptsize{(0.02)} \\ 
			\texttt{WORSE}   & 0.00 \scriptsize{(0.00)} & 0.00 \scriptsize{(0.00)} & 0.00 \scriptsize{(0.00)} \\ 
			\texttt{NONE}    & 0.85 \scriptsize{(0.01)} & 0.92 \scriptsize{(0.01)} & 0.88 \scriptsize{(0.01)} \\ 
			average & 0.73 \scriptsize{(0.01)} & 0.79 \scriptsize{(0.01)} & 0.76 \scriptsize{(0.01)} \\ 
			\bottomrule
		\end{tabular}
	\end{table}
	
		\begin{table}[h] 
		\centering 
		\caption{ Binary feature which represents the presence of a comparative adjective in the sentence (three-class scenario). } 
		\label{  }
		\begin{tabular}{@{}lrrrr@{}}
			\toprule
			        & precision                & recall                   & f1 score                 \\ \midrule 
			\texttt{ARG}     & 0.75 \scriptsize{(0.02)} & 0.53 \scriptsize{(0.02)} & 0.62 \scriptsize{(0.02)} \\ 
			\texttt{NONE}    & 0.84 \scriptsize{(0.01)} & 0.93 \scriptsize{(0.01)} & 0.89 \scriptsize{(0.01)} \\ 
			average & 0.82 \scriptsize{(0.01)} & 0.82 \scriptsize{(0.01)} & 0.81 \scriptsize{(0.01)} \\ 
			\bottomrule
		\end{tabular}
	\end{table}
	
	
	
	\begin{table}[h] 
		\centering 
		\caption{LexNet path embeddings with a maximum length of four and restrictions of the edge direction (three-class scenario). This setup is equal to the original setup in \cite{DBLP:journals/corr/ShwartzD16}} 
		\label{  }
		\begin{tabular}{@{}lrrrr@{}}
			\toprule
			        & precision                & recall                   & f1 score                 \\ \midrule 
			\texttt{BETTER}  & 0.65 \scriptsize{(0.02)} & 0.20 \scriptsize{(0.01)} & 0.30 \scriptsize{(0.02)} \\ 
			\texttt{WORSE}   & 0.58 \scriptsize{(0.12)} & 0.07 \scriptsize{(0.02)} & 0.13 \scriptsize{(0.03)} \\ 
			\texttt{NONE}    & 0.76 \scriptsize{(0.00)} & 0.98 \scriptsize{(0.00)} & 0.86 \scriptsize{(0.00)} \\ 
			average & 0.73 \scriptsize{(0.01)} & 0.76 \scriptsize{(0.00)} & 0.69 \scriptsize{(0.00)} \\ 
			\bottomrule
		\end{tabular}
	\end{table}
	
	\begin{table}[h] 
		\centering 
		\caption{LexNet path embeddings with a maximum length of four and restrictions of the edge direction (binary scenario). This setup is equal to the original setup in \cite{DBLP:journals/corr/ShwartzD16}} 
		\label{  }
		\begin{tabular}{@{}lrrrr@{}}
			\toprule
			        & precision                & recall                   & f1 score                 \\ \midrule 
			\texttt{ARG}     & 0.78 \scriptsize{(0.06)} & 0.21 \scriptsize{(0.02)} & 0.33 \scriptsize{(0.02)} \\ 
			\texttt{NONE}    & 0.77 \scriptsize{(0.00)} & 0.98 \scriptsize{(0.01)} & 0.86 \scriptsize{(0.01)} \\ 
			average & 0.77 \scriptsize{(0.02)} & 0.77 \scriptsize{(0.01)} & 0.72 \scriptsize{(0.01)} \\ 
			\bottomrule
		\end{tabular}
	\end{table}
	
	%===
	\begin{table}[h] 
		\centering 
		\caption{LexNet path embeddings with a maximum length of sixteen and no restrictions of the edge direction (three-class scenario).} 
		\label{  }
		\begin{tabular}{@{}lrrrr@{}}
			\toprule
			        & precision                & recall                   & f1 score                 \\ \midrule 
			\texttt{BETTER}  & 0.64 \scriptsize{(0.05)} & 0.64 \scriptsize{(0.03)} & 0.64 \scriptsize{(0.02)} \\ 
			\texttt{WORSE}   & 0.41 \scriptsize{(0.08)} & 0.15 \scriptsize{(0.04)} & 0.22 \scriptsize{(0.04)} \\ 
			\texttt{NONE}    & 0.87 \scriptsize{(0.01)} & 0.93 \scriptsize{(0.01)} & 0.90 \scriptsize{(0.00)} \\ 
			average & 0.79 \scriptsize{(0.01)} & 0.81 \scriptsize{(0.01)} & 0.80 \scriptsize{(0.01)} \\ 
			\bottomrule
		\end{tabular}
	\end{table}
	
	\begin{table}[h] 
		\centering 
		\caption{LexNet path embeddings with a maximum length of sixteen and no restrictions of the edge direction (binary scenario).} 
		\label{  }
		\begin{tabular}{@{}lrrrr@{}}
			\toprule
			        & precision                & recall                   & f1 score                 \\ \midrule 
			\texttt{ARG}     & 0.75 \scriptsize{(0.01)} & 0.65 \scriptsize{(0.02)} & 0.70 \scriptsize{(0.01)} \\ 
			\texttt{NONE}    & 0.88 \scriptsize{(0.00)} & 0.92 \scriptsize{(0.00)} & 0.90 \scriptsize{(0.00)} \\ 
			average & 0.84 \scriptsize{(0.00)} & 0.85 \scriptsize{(0.00)} & 0.84 \scriptsize{(0.00)} \\ 
			\bottomrule
		\end{tabular}
	\end{table}
	
\section{Final Held-Out Experiments}

%\include{kapitel5}
%\include{kapitel6}
%\include{kapitel7}
\cleardoublepage

% VERZEICHNISSE (Abbildungen, Tabellen)
% Literatur 
\bibliographystyle{apalike}
\bibliography{mathesis}
\cleardoublepage

% ERKLÄRUNG
\chapter*{Eidesstattliche Versicherung}
\thispagestyle{empty}
\addcontentsline{toc}{chapter}{Eidesstattliche Versicherung}

Hiermit versichere ich an Eides statt, dass ich die vorliegende Arbeit im Masterstudiengang Informatik selbstständig verfasst und keine anderen als die angegebenen Hilfsmittel – insbesondere keine im Quellenverzeichnis nicht benannten Internet-Quellen – benutzt habe. Alle Stellen, die wörtlich oder sinngemäß aus Veröffentlichungen entnommen wurden, sind als solche kenntlich gemacht. Ich versichere weiterhin, dass ich die Arbeit vorher nicht in einem anderen Prüfungsverfahren eingereicht habe und die eingereichte schriftliche Fassung der auf dem elektronischen Speichermedium entspricht.

\noindent Ich bin mit einer Einstellung in den Bestand der Bibliothek des Fachbereiches einverstanden.

\vspace{2cm} 

\noindent Hamburg, den
    
\end{document}